%Title: Beamer Presentation Template
%Author: LISA
%Year: 2020

\documentclass[10pt,aspectratio=169]{beamer}

 
 
%
%Setting file
%

\usepackage[T1]{fontenc}
\usepackage[utf8]{inputenc}

\usepackage[english]{babel}

\usepackage{graphicx}
\graphicspath{{images/}}
\usepackage{float}
\usepackage{tikz}
\usepackage{caption}
\usepackage{subcaption}
\usepackage[absolute,overlay]{textpos}
\usepackage{mathtools}

\usetheme{default}
\usefonttheme{structurebold}


% ---------------------------------
% color definitions
\usepackage{color}
\definecolor{LISA_BLUE}{cmyk}{0.99,0.88,0.29,0.18}
\setbeamercolor{normal text}{fg=LISA_BLUE}
\setbeamercolor{frametitle}{fg=LISA_BLUE}

\colorlet{itemizecolor}{red}

\newcommand\insertlocation{}  % Empty by default.
\newcommand\location[1]{\renewcommand\insertlocation{#1}}

\newcommand\insertperiod{}  % Empty by default.
\newcommand\period[1]{\renewcommand\insertperiod{#1}}



\setbeamertemplate{itemize items}[circle]
\setbeamercolor{title}{fg=white}

\setbeamertemplate{title page}
{
	\begin{textblock*}{0.6\paperwidth}(0.05\paperwidth,0.25\paperheight)
         

      \usebeamerfont{title}\textcolor{white}{\inserttitle}%
      \ifx\insertsubtitle\@empty%
      \else%
        \vskip0.25em%
        {\usebeamerfont{subtitle}\usebeamercolor[fg]{subtitle}\insertsubtitle\par}%
      \fi%     

      \vspace{0.1\textheight}      
      \usebeamerfont{author}\textcolor{white}{\insertauthor}
      \vskip0.25em%
      \usebeamerfont{institute}\textcolor{white}{\insertinstitute}

  \end{textblock*}
  
}

\makeatletter
    \newenvironment{SectionTitle}{
        % \setbeamertemplate{headline}[default]
        \setbeamercolor{normal text}{fg=LISA_BLUE}
        \setbeamercolor{frametitle}{fg=LISA_BLUE}
        \setbeamercolor{title}{fg=LISA_BLUE}
        \setbeamercolor{normal text}{fg=LISA_BLUE}
        \usebeamercolor[fg]{normal text}
        \usebeamercolor[fg]{footline}
        \usebackgroundtemplate%
        {%
            \includegraphics[width=\paperwidth,height=\paperheight]{Section.pdf}%
        }

    }{}
\makeatother

\useoutertheme{smoothbars} 
\setbeamertemplate{footline}[frame number]{}
\setbeamertemplate{navigation symbols}{}
\setbeamertemplate{footline}{}


\usebackgroundtemplate%
{%
    \includegraphics[width=\paperwidth,height=\paperheight]{Slide.pdf}%
}


\setbeamertemplate{footline}
{
 \begin{beamercolorbox}[wd=\paperwidth, ht=0.05\paperheight, dp=5.2pt,leftskip=2cm,rightskip=.3cm]{section in head/foot}
 \hspace{0.095cm}\textcolor{white}{\insertdate \hfill  \insertframenumber}\phantom{x}\vskip2pt
 \end{beamercolorbox}%
}

\makeatletter
\setbeamertemplate{frametitle}{
  \vspace{0.05\textheight}
  \begin{beamercolorbox}[center]{frametitle}
      \usebeamerfont{frametitle} \insertframetitle%
  \end{beamercolorbox}%
}
\makeatother
% \beamertemplatenavigationsymbolsempty
% \setbeamercovered{transparent}

\setbeamertemplate{headline}
{%

}


%-----------------------------------------Title page settings-----------------------------------------%
\title{ LISA Specialized Training~4}
\subtitle{\small Advanced Computational techniques}
\author{\small J. Warbinek$^{1,2}$, L. Reed$^{3}$, A. Raggio$^{4}$}
\institute{\tiny
	$^{1}$GSI Helmholtzzentrum für Schwerionenforschung, Darmstadt, Germany\\
	$^{2}$Helmoltz Institute Mainz, Mainz, Germany\\
	$^{3}$Department Chemie - Standort TRIGA, Johannes Gutenberg - Universit\"{a}t Mainz, Germany\\
	$^{4}$Department of Physics, University of Jyv\"{a}skyl\"{a}, Finland\\
		

}
% \period{Month-20XX}
% \location{Event}


%-----------------------------------------Title page settings-----------------------------------------%


\begin{document}

{
  \usebackgroundtemplate{\includegraphics[width=\paperwidth]{Title.pdf}}
	\begin{frame}[noframenumbering, plain]
		\titlepage
	\end{frame}
}

\begin{SectionTitle}
\begin{frame}
	\begin{textblock*}{0.5\paperwidth}(0.25\paperwidth,0.25\paperheight)
		\centering
		\textbf{\LARGE Nuclear DFT\\ {\large Shapes and radii}}
	\end{textblock*}
	\begin{textblock*}{0.5\paperwidth}(0.25\paperwidth,0.5\paperheight)
		\centering
		\includegraphics[width=.8\textwidth]{Shapes.pdf}
	\end{textblock*}

\end{frame}
\end{SectionTitle}

\begin{frame}{SLIDE 1}
	An empty fancy slide.
\end{frame}

\begin{frame}{SLIDE 2}
	An empty fancy slide.
\end{frame}

\begin{SectionTitle}
	\begin{frame}
		\begin{textblock*}{0.5\paperwidth}(0.25\paperwidth,0.25\paperheight)
			\centering
			\textbf{\LARGE Ion Beam Simulation}
		\end{textblock*}
		\begin{textblock*}{0.5\paperwidth}(0.25\paperwidth,0.4\paperheight)
			\centering
			\includegraphics[width=.8\textwidth]{ibs_logo.pdf}
		\end{textblock*}
	
	\end{frame}
 \end{SectionTitle}


\begin{frame}{SIMION Simulations}
    \begin{textblock*}{0.5\paperwidth}(0.5\paperwidth,0.4\paperheight)
        \centering
        \begin{itemize}
            \item Flying 1000 ions
            \item Circle distribution start point along x-axis
            \item Cone direction distribution (5°) along z-axis
            \item Kinetic energy 10keV, acceleration of 10keV
            \item Extract $x_i$, $x_i'$, $x_f$, $x_f'$ for each ion
        \end{itemize}
    \end{textblock*}
    \begin{textblock*}{0.5\paperwidth}(0.02\paperwidth,0.3\paperheight)
			\centering
			\includegraphics[width=.9\textwidth]{Exercises_presentation/images/SimionScreenshot.png}
		\end{textblock*}
  
\end{frame}

\begin{frame}{Description of Transfer Matrix}
    \begin{textblock*}{0.5\paperwidth}(0.425\paperwidth,0.2\paperheight)
        \centering
        \begin{equation*}
            \begin{pmatrix}x_f \\ x_f'\end{pmatrix} = M_\text{drift  } M_\text{gap  } M_\text{drift  } \cdot \begin{pmatrix}x_i \\ x_i'\end{pmatrix}
        \end{equation*}
        
        \begin{equation*}
            \begin{pmatrix}x_f \\ x_f'\end{pmatrix} = \begin{pmatrix} a_{11} & a_{12}\\  a_{21} & a_{22}\end{pmatrix} \cdot \begin{pmatrix}x_i \\ x_i'\end{pmatrix}
        \end{equation*}

        \begin{equation}
            x_f = a_{11} x_i +  a_{12} x_i  
        \end{equation}
        \begin{equation}
            x_f' = a_{21} x_i +  a_{22} x_i 
        \end{equation}
            
    \end{textblock*}
    \begin{textblock*}{0.5\paperwidth}(0.\paperwidth,0.35\paperheight)
			\centering
			\includegraphics[width=.8\textwidth]{ibs_logo.pdf}
		\end{textblock*}
  
\end{frame}

\begin{frame}{Extraction of Transfer Matrix Elements $a_{ij}$}
    \begin{textblock*}{0.5\paperwidth}(0.02\paperwidth,0.25\paperheight)
			\centering
			\includegraphics[width=.9\textwidth]{Exercises_presentation/images/FirstCoord_Eq.pdf}
            \begin{equation*}
                \frac{x_f}{x_i'} = a_{11}\frac{x_i}{x_i'} + a_{12}
            \end{equation*}
		\end{textblock*}
    \begin{textblock*}{0.5\paperwidth}(0.52\paperwidth,0.25\paperheight)
			\centering
			\includegraphics[width=.9\textwidth]{Exercises_presentation/images/SecondCoord_Eq.pdf}
            \begin{equation*}
                \frac{x_f'}{x_i} = a_{22}\frac{x_i'}{x_i} + a_{12}
            \end{equation*}
		\end{textblock*}
  
\end{frame}


\begin{frame}{Transfer Matrix Calculation}
    \begin{textblock*}{0.5\paperwidth}(0.425\paperwidth,0.2\paperheight)
        \centering
        \begin{equation*}
            \begin{pmatrix}x_f \\ x_f'\end{pmatrix} = M_\text{drift  } M_\text{gap  } M_\text{drift  } \cdot \begin{pmatrix}x_i \\ x_i'\end{pmatrix}
        \end{equation*}
        Drift matrix $M_\text{drift  }$: 
        \begin{equation*}
            \begin{pmatrix}x_f \\ x_f'\end{pmatrix} \xrightarrow{\text{Drift length tube S}} \begin{pmatrix} x_f + x_f' S \\ x_f'\end{pmatrix}
        \end{equation*}

        \begin{equation*}
            M_\text{drift  } = \begin{pmatrix} 1 & S \\ 0 & 1 \end{pmatrix}, \quad  M_\text{drift  }^{-1} = \begin{pmatrix} 1 & -S \\ 0 & 1 \end{pmatrix}
        \end{equation*}

            
    \end{textblock*}
    \begin{textblock*}{0.5\paperwidth}(0.\paperwidth,0.35\paperheight)
		\centering
		\includegraphics[width=.8\textwidth]{Exercises_presentation/images/ibs_logo_2.pdf}
	\end{textblock*}

\end{frame}

\begin{frame}{Transfer Matrix Calculation}
    \begin{textblock*}{0.5\paperwidth}(0.425\paperwidth,0.27\paperheight)
        \centering
        Result from fit to simulation Ddata:
        \begin{equation*}
            M_\text{drift  } M_\text{gap  } M_\text{drift  } = \begin{pmatrix} 1.022 & 0.151 \\ \approx 0 & 0.707 \end{pmatrix}
        \end{equation*}

        with drift Length $S=50$\,mm

         \vspace{0.1\paperheight}
        
        Multiplying with $M_\text{drift  }^{-1}$ from both sides yields: 

        \begin{equation*}
            M_\text{gap  } = \begin{pmatrix} 1.022 & -86.310  \\ \approx 0 & 0.707 \end{pmatrix}
        \end{equation*}

            
    \end{textblock*}
    \begin{textblock*}{0.5\paperwidth}(0.\paperwidth,0.35\paperheight)
		\centering
		\includegraphics[width=.8\textwidth]{Exercises_presentation/images/ibs_logo.pdf}
	\end{textblock*}
  

\end{frame}

\end{document}